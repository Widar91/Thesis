\chapter{Appendix A}
\label{app:a}

In the following sections we are going to dive deeper in the theoretical notions that have powered the derivation in this thesis. A special mention of Steve Awodey's book \textit{"Category Theory"}\cite{awodey2010category} and the HaskellWiki\cite{haskellwiki} for providing much inspiration in the exposition of these topics is in order.

\section{Duality - Category Theory}

Duality is an element of paramount importance in research we presented. Up until this moment we have only dealt with an informal definition of duality but its roots go very deep in the field of mathematics and category theory in particular.

Given the formal definition of a category, composed of objects $$A,\; B,\; C,\; ...$$ and arrows $$f,\; g,\; h,\; ... $$ and operations $$ dom(f) \quad cod(f) \quad id(A) \quad g \circ f $$ 

Given any sentence $ \Sigma $, we can create its dual $ \Sigma^{op} $ by interchanging $$dom \to cod$$ and the order of composition $$ g \circ f \to f \circ g $$ 

It is clear how $ \Sigma^{op} $ is a well formed sentence in the language of category theory $ CT $.

The principle of - formal - duality then tells us that for any sentence $ \Sigma $ in the language of category theory, if $ \Sigma $ follows from the axioms defined for categories, then so does $ \Sigma^{op}$. 

$$ CT \vdash \Sigma \implies CT \vdash \Sigma^{op} $$

From a visual perspective, this definition boils down to reversing the order of composition of the arrows. 

\begin{center}
\begin{minipage}{.3\textwidth}
\begin{tikzpicture}
\large
  \matrix (m) [matrix of math nodes,row sep=3em,column sep=4em,minimum width=2em]
  { A & B \\
       & C \\};
  \path[-stealth]
    (m-1-1) edge node [above] {$f$} (m-1-2)
                 edge node [below] {$g \circ f\qquad$} (m-2-2)
    (m-1-2) edge node [right] {$g$} (m-2-2);
\end{tikzpicture}
\end{minipage}
\begin{minipage}{.3\textwidth}
\begin{tikzpicture}
\large
  \matrix (m) [matrix of math nodes,row sep=3em,column sep=4em,minimum width=2em]
  { A & B \\
       & C \\};
  \path[-stealth]
    (m-1-2) edge node [above] {$f$} (m-1-1)
    (m-2-2) edge node [below] {$f \circ g\qquad$} (m-1-1)
                 edge node [right] {$g$} (m-1-2);
\end{tikzpicture}
\end{minipage}
\end{center}

\section{Products \& Coproducts}

In category theory, the product of two objects in a category is the most general object which admits a morphism to each of the ones that compose it. The notion of product aims at capturing the essence of more specialized definitions found in various categories and areas of mathematics. The easiest way understand this construct is to start from the cartesian products in the category of sets.

Given sets $A$ and $B$, let us define the cartesian product as the set $$A \times B\; =\; \{\; (a, b)\; |\; a \in A,\; b \in B\; \} $$ There are two coordinate projections
\begin{center}
\begin{tikzpicture}
  \matrix (m) [matrix of math nodes,column sep=4em,minimum width=2em] { 
	  A & {A \times B} & B \\
	  & & \\
  };
  \path[-stealth]
    (m-1-2) edge node [above] {$fst$} (m-1-1)
                 edge node [above] {$snd$} (m-1-3);
\end{tikzpicture}
\end{center}
where $$ fst\; (a, b) = a \qquad snd\; (a, b) = b $$ 

It follows that given any element $ c \in A \times B $ $$ c\; =\; (\; fst\; (c),\; snd\; (c)\; )$$ 
The following diagram captures the essence of cartesian products.

\begin{center}
\begin{tikzpicture}
  \matrix (m) [matrix of math nodes,row sep=7em,column sep=6em,minimum width=2em] { 
	 {} & 1 & {} \\
	  A & {A \times B} & B \\
  };
  \path[-stealth]
  	(m-1-2) edge node [left] {$a\quad$} (m-2-1)
                 edge node [right] {$\quad b$} (m-2-3)
                 edge node [right] {$(a, b)$} (m-2-2)
    (m-2-2) edge node [below] {$fst$} (m-2-1)
                 edge node [below] {$snd$} (m-2-3);
\end{tikzpicture}
\end{center}

The definition of categorical products can be derived from generalizing the elements in the previous definition.

Coproducts are the dual notion of categorical products, representing the least general object to which the objects in the family admit a morphism. Within the context of set theory, the represent the disjoint union of sets.

Given sets $A$ and $B$, let us define their disjoint union as the set $$A + B\; =\; \{\; (a, 1)\; |\; a \in A\} \cup (b, 2)\; |\; b \in B\} $$ There are two injection functions
\begin{center}
\begin{tikzpicture}
  \matrix (m) [matrix of math nodes,column sep=4em,minimum width=2em] { 
	  A & {A + B} & B \\
	  & & \\
  };
  \path[-stealth]
    (m-1-1) edge node [above] {$left$} (m-1-2)
    (m-1-3) edge node [above] {$right$} (m-1-2);
\end{tikzpicture}
\end{center}
where $$ left\; (a) = (a,\; 1) \qquad right\; (b) = (b,\; 2) $$ 

The essence is captured by the following diagram, where we have simply reversed the arrows.

\begin{center}
\begin{tikzpicture}
  \matrix (m) [matrix of math nodes,row sep=7em,column sep=6em,minimum width=2em] { 
	 {} & 1 & {} \\
	  A & {A + B} & B \\
  };
  \path[-stealth]
  (m-2-1) edge node [left] {$f\quad$} (m-1-2)
  (m-2-3) edge node [right] {$\quad g$} (m-1-2)
  (m-2-2) edge node [right] {$[f,g]$} (m-1-2)
  (m-2-1) edge node [below] {$left$} (m-2-2)
  (m-2-3) edge node [below] {$right$} (m-2-2);
\end{tikzpicture}
\end{center}

where $$ [f, g](x,\delta) = \begin{cases} f(x), & \delta = 1 \\ g(x), & \delta = 2 \end{cases} $$

Withing the context of our research and the Haskell programming language, tuples represent products and the Either monad represents coproducts.

\section{Curry \& Uncurry}

Currying is the technique of transforming a function taking multiple arguments as input in one taking only the first argument and returning a function that takes the remainder of the arguments and returns the result of the initial function. Uncurrying is the opposite of currying, taking a curried function into one that accepts multiple arguments as input.

\hminted
curry   :: ((a, b) -> c) -> (a -> b -> c)
uncurry :: (a -> b -> c) -> (a, b) -> c

f :: a -> b -> c
g :: (a, b) -> c

f = curry g
g = uncurry f

-- f x y = g (x, y)
-- curry . uncurry = id
\end{minted}

\section{Functors}

In category theory a functor is a mapping from one category to another or a homomorphism of categories where certain laws hold. 

Functors span a large number of categories and are more and more common in modern programming languages. In Haskell the Functor class and its laws are defined as follows.
\hminted
class Functor f where
    fmap :: (a -> b) -> f a -> f b
  
-- fmap id      = id
-- fmap (p . q) = (fmap p) . (fmap q)
\end{minted}

\section{Continuation Passing Style}

\textit{Continuation Passing Style (CPS)}, is a style of programming where functions do not return their result type directly but forward it to an extra function called \textit{continuation}, which will specify what will happen next in the control flow of the program. 

In CPS, functions take the role of suspended computations and effectively do nothing until a continuation is passed in as an argument.

The main advantage of this programming style is the power of controlling and altering the control flow of a program to such a great extent that it is possible, with continuations, to implement features such as exceptions and concurrency.

The Haskell programming language exposes continuations through the Continuation Monad, also known as the mother of all monads, as its definition makes it possible to implement any other monad.

\haskellcode{src/continuation_monad.hs}




\
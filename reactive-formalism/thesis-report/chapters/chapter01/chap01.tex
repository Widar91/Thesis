\let\textcircled=\pgftextcircled
\chapter{Reactive Programming}
\label{chap:reactiveprogramming}

\epigraph{\hspace{4ex}\textit{The cold winds are rising in the North... Brace yourselves, winter is coming.}}{--- George R.R. Martin,\\ \textit{A Game of Thrones}}


In this chapter we are going to introduce the concept of reactive programming and motivate it's importance and relevance with regards to modern applications and the type of problems developers have to face nowadays. We are then going to introduce the most popular commercial libraries that claim to solve the reactive problem, with the purpose of giving the reader some context for our discussion and motivating the need for a mathematical formalization that abstracts over the class of problems these implementations set out to address.

\section{The Essence of Reactive Programs}

The use of the term reactive program in scientific literature is dated back to the mid-sixties\cite{scopus-reactive}. A relevant and insightful definition was given by G. Berry in 1991\cite{berry1991reactive} as he describes reactive programs in relation to their dual counterparts, interactive programs:

\begin{quote}
\hspace{4ex}``\textit{Interactive programs} interact at their own speed with users or with other programs; from a user point of view, a time-sharing system is interactive. \textit{Reactive programs} also maintain a continuous interaction with their environment, but at a speed which is determined by the environment, not by the program itself.''
\end{quote}

Interactive programs concretize the idea of a pull-based model of computation, where the program - the consumer in this case - has control over the speed at which data will be requested and handled. A perfect example of an interactive program is a control-flow structure such as a for-loop iterating over a collection of data: the program is in control of the speed at which data is retrieved from the containing collection and will request the next element only after it is done handling the current one.

Reactive programs, on the contrary, embody the idea of a push-based - or event-driven - model of computation, where the speed at which the program interacts with the environment is determined by the environment rather than the program itself. In other words, it is now the producer of the data - i.e. the environment - who determines the speed at which events will occur whilst the program's role reduces to that of a silent observer that will react upon receiving events. Standard example of such systems are GUI applications dealing with various events originating from user input - e.g. mouse clicks, keyboard button presses - and programs dealing with stock markets, social media or any other kind of asynchronous updates.  

\section{Why Reactive Programming Matters}
\label{sec:whyrpmatters}

Considering the definition and examples of reactive programs we analyzed in the previous section, let's now try to formalize the class of problems the reactive programming paradigm is specifically well-suited for.

The table below provides a collection of types offered by common programming languages for handling data, parameterized over two variables: the size of the data, either one or multiple values, and the way data is handled, either by synchronous or asynchronous computations\cite{meijer2015spicing}.

\begin{center}
    \begin{tabular}{| l | l | l |}
    \hline
    & \textbf{One} & \textbf{Many} \\ 
    \hline
	\textbf{Sync} & \code{a} & \code{Iterable a} \\ 
	\hline
	\textbf{Async} & \code{Future a} & \textit{Reactive Programming} \\ 
	\hline
    \end{tabular}
    %\caption{The four fundamental effects}
\end{center}

The first row shows that synchronous functions come in two flavors: classic functions that return a single value of type \code{a} and functions that produce a collection of results of type \code{a}, abstracted through the \code{Iterable a} interface (See section \ref{sec:iterables}). These types of functions embody the standard imperative, pull-based approach to programming, where a call to a function/method synchronously blocks until a result is produced. 

Moving on to the second row, we encounter \code{Future a}, an interface representing an asynchronous computation that, at a certain point in the future, will result in a value of type \code{a}. Futures are generally created by supplying two callbacks together with the asynchronous computation, one to be executed in case of success and the other one in case of error. 

Programming languages, however, are not as well equipped when it comes to handling asynchronous computations resulting in multiple values - i.e. push-based collections. The issue lies in the fact that the program's control flow is dictated by the environment rather than the program itself - i.e. inversion of control -, making it very hard to model such problems with commonly known control structures, which are optimized for sequential models of computation. Traditional solutions typically involve developers manually trying to compose callbacks by explicitly writing CPS (continuation passing style) code\cite{meijer2015spicing}, resulting in what it's commonly referred to as \textit{Callback Hell}\cite{edwards2009coherent}.

The aforementioned class of problems reflects the definition of reactive programs we analyzed in the previous section, where the environment asynchronously - i.e. at its own speed - pushes multiple events to the program. The reactive programming paradigm sets out to provide interfaces and abstractions to facilitate the modeling of such problems as push-based collections. 

\section{Related Work}

Interfaces are only as good as the implementations that back them up. In this section we are going to discuss and analyze the most commonly known APIs and libraries that claim to embody the reactive paradigm, motivating our need for a mathematical formalization to aid in unifying these different approaches under a single set of interfaces. For a more extensive discussion of reactive programming libraries, including 


The most commonly known attempt at defining reactive programming is the Reactive Manifesto\cite{reactive-manifesto}, a document that tries to describe reactive systems in terms of design principles and conceptual architecture. Although certainly insightful, we feel the document targets a more management-focused audience and certainly lacks any type of scientificity. 

\todo[inline]{Talk about FRP, Reactive Streams, Rx and other stuff. CIte survey paper
Claim that Observable is dual of Iterator in order to justify the why.}

\subsection{Reactive Extensions}
\label{subsec:rx}

Reactive Extensions - also known as Rx - is the standard library for programming in a reactive way. Originally published by Microsoft as an API for the C\# and Javascript languages, it was later ported to the JVM world by Netflix as an open source project, gaining much traction in the developers community and resulting in various implementations for the currently most commonly used programming languages\cite{ReactiveX}.

The intuition and theory on which Reactive Extensions are built originated from the mind of Erik Meijer\cite{meijer2010observable} and will be at the basis of the work developed in this thesis, where we will use mathematical constructs and derivations in order to prove the correspondence between the interfaces exposed by this library and the essence of reactive programming we will derive.

Although originally based on theoretically sound concepts, this polyglot family of libraries diverged from a purely reactive implementation, mainly due to their open source nature, independent development and, most importantly, to the lack of a unifying reference formalization of the reactive paradigm. This last aspect further motivates the need for our research.

Rx defines itself as a a library for composing asynchronous and event based (reactive) programs by using observable sequences\cite{Rx.Net}. At its core, it expose two interfaces, \code{Observable} and \code{Observer}. An \code{Observable} is the producer of a sequence of events which are pushed to an \code{Observer}, who will act upon them and produce side effects. Furthermore, the library offers a number of additional constructs such as \code{Subscription}, \code{Scheduler} and operators, that facilitate programming with asynchronous events and make the API more appealing for use in a production environment.

\subsection{Reactive Streams}

bla\cite{meijer2014reactive}
Try to define itself as reactive programming, mathematics prove the claim wrong, they implement asynchronous iterables. This does not make them useless, yet they are not a good solution for the problem we are trying to solve.

\todo[inline]{Probably put the derivation of Reactive Streams in the appendix, it's not really the focus, I guess.}

\subsection{Functional Reactive Programming}
\label{frp}
Much researched topic, beautiful in theory, yet based on continuous time, un-achievable with computers where time is inherently discrete. Connel Elliot, the inventor of FRP, says that commercial implementations are far from the original theory and do not really represent the essence of FRP
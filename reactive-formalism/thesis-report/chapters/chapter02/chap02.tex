\let\textcircled=\pgftextcircled
\Chapter{Into the Rabbit Hole}{Deriving the Observable}

\epigraph{\hspace{4ex}\textit{"It was much pleasanter at home," thought poor Alice, "when one wasn't always growing larger and smaller, and being ordered about by mice and rabbits. I almost wish I hadn't gone down the rabbit-hole -- and yet -- and yet -- ..."}}{--- Lewis Carol,\\ \textit{Alice in Wonderland}}

As we saw in Chapter \ref{chap:reactiveprogramming}, the \code{Iterable} interface embodies the idea of a pull-based model of computation and is the commonly adopted solution to dealing with synchronous computations resulting in multiple values. In this chapter we are going to formalize the intuition that there exists a duality relation between interactive and reactive programs, as well as between pull and push models of computations, by deriving the \code{Observable} interface starting from its dual counterpart, the \code{Iterable}.

The derivation that follows will require the use of a number mathematical concepts such as \textit{categorical duality}, \textit{continuations}, \textit{(co)products}, \textit{(un)currying}, \textit{covariance}, \textit{contravariance} and \textit{functors}. We suggest the reader to get familiar with these topics before diving into the derivation. An accessible introduction to each can be found in Appendix \ref{app:a}.

\section{Iterables}
\label{sec:iterables}

An \code{Iterable} is a programming interface which enables the user to traverse a collection of data, abstracting over the underlying implementation\cite{gamma1995design}.

The interface and semantics of \code{Iterable}s were first introduced by the Gang of Four though their Iterator pattern\cite{gamma1995design}; today's most used programming languages introduce the \code{Iterable} as the root interface for standard pull collections APIs - exposing concrete implementations such as maps, sets, indexed sequences and so on. 

The \code{Iterable} interface is generally fixed across programming languages, with the exception of naming conventions - C\# and related languages call it IEnumerable - and slight differences in the types, as we can see from the following definitions.

\haskellcode{src/iterable_interface.hs}

Although the essence of the pattern is preserved by both definitions, we claim that the C\# version more clearly and accurately reflects the way side effects play a role in the usage of the interface: \code{moveNext} contains all the side effects of traversing the underlying collection and retrieving the next value while \code{current} can inspect the retrieved value multiple times in a pure way. The Java version, on the other hand, embeds the side effect in the \code{next} function, making it impossible to inspect the current value multiple times. For this reason and without loss of generality, we will make use of the C\# definition - modulo naming conventions - in the reminder of the discussion:

\haskellcode{src/iterable_interface_final.hs}

\section{The Essence of Iterables}

The first step in deriving the \code{Observable} is to simplify our \code{Iterable} definition to a type that reflects its very essence; we are gonna do this by stripping the interface presented in the previous section of all the unnecessary operational features that only clutter our definition.

Let's start by taking a closer look at the \code{Iterator} interface; we can observe that the definition of the functions \code{moveNext} and \code{current} is equivalent to a single function which returns either a value - analogous to a \code{moveNext} call returning true and a subsequent invocation to \code{current} - or nothing - analogous to a call to \code{moveNext} returning false. 

Before we formalize this observation with a proper type, let us notice another effect that is hidden in the current definition of \code{moveNext} and not made explicit by the its type: the possibility for an exception to be thrown by the function's body. 

By merging these considerations with the notion of coproducts and Haskell's \code{Either} and \code{Maybe} type, we obtain the following definition.

\haskellcode{src/iterable_coproducts_maybe.hs}

Note how, theoretically, \code{getIterator} could also throw an exception. We assume here, without loss of generality, that the function will never throw and will always be able to return an \code{Iterator} instance.

The next step is to forget about data types and express our interfaces as simple types.

\haskellcode[firstline=1,lastline=2]{src/iterable_types.hs}

At this point, we want to put aside the operational concerns regarding exceptions and termination and assume the \code{Iterator} function will always return a value of type \code{a}.

\haskellcode[firstline=4,lastline=5]{src/iterable_types.hs}

We have now reached a point where no simplification is possible anymore. The obtained types reflect the essence of the Iterator patter: an \code{Iterable} is, theoretically, a function which, when invoked, produces an \code{Iterator} and an \code{Iterator} is itself a function producing a value of type \code{a} as a side effect. 

When looking at the \code{Iterator} type from an object oriented perspective, the reader should notice a strict similarity to a \textit{getter} function - i.e. a lazy producer of values: iterators are, in fact, nothing more than getters of values of type \code{a}. The \code{Iterable}, on the other hand, is a function that enables the user to get an \code{Iterator}, i.e. a getter of a getter of \code{a}.

This correspondence will turn out to be very insightful later on in our discussion, where we will observe that \code{Observable} is nothing more that a setter of setters, another instance of duality in our formalization. 

When looking at the relation between the \code{Iterator} type and its base component, \code{a}, we can observe how they are bound by a covariant relation:

\hminted
                                   A <: B
                      ----------------------------------
                          () -> IO A <: () -> IO B
\end{minted}

The intuition can be easily understood when we think of an iterator as a drink vending machine, i.e. a function which, whenever called, will give back a drink:

\hminted
                                Coke <: Drink
                      ---------------------------------
                       VendingM Coke <: VendingM Drink
\end{minted}

If coke is a subtype of drink, then whenever I am asked for a drink vending machine, I can hand out a coke vending machine without incurring in any troubles with the person who asked, as that machine will correctly provide drinks - even though they will always be coke - whenever prompted for one.

With \code{Iterator} being a getter itself, it should be clear how covariance plays the same role as with \code{Iterable}.

To formally prove the intuition of a covariant relation, we instantiate the \code{Iterable}/\code{Iterator} types to a covariant \code{Functor} and prove its laws in the following snippet.

\haskellcode{src/iterable_functor.hs}

For the sake of completeness, it is worth mentioning that \code{Iterable} is, among others, also an instance of Applicative Functor and Monad. Although certainly interesting from a theoretical perspective, showing these instances and proving the associated laws goes beyond the scope of this work. Nontheless, we will see in the next section how these concepts are relevant in expressing and motivating the duality between Iterables and Observables.

\section{Applying Duality}
\label{sec:applyingduality}

By now, the reader should be somehow familiar with the concept of duality, as it has appeared many times throughout our discussion in concepts such as pull and push models of computation or interactive and reactive programs. Duality is, in fact, a very important general theme that has manifestations in almost every area of mathematics\cite{gowers2010princeton} (See Appendice \ref{app:a} for an introductory discussion on the topic). 

Starting from the fact that the \code{Iterable} interface embodies the idea of interactive programming, let's use the principle of duality to derive the \code{Observable} interface and see how it relates to the concept of reactive programming. In practice, this translates to the simple task of flipping the function arrows in the \code{Iterable} interface, taking us from a function resulting in a value of type \code{a} to one accepting an \code{a}.

\haskellcode{src/iter_to_obs.hs}

Note how the side effects are bound to function application rather than values, hence their flipped position in the \code{Observable} type.

The newly derived types are relatively easy to read and understand: \code{Observer} is simply a function that, given a value of type \code{a} will handle it somehow, producing side effects; the \code{Observable}, on the other hand, is responsible for producing such values of type \code{a} and feeding them to the \code{Observer} it has been given as an argument.

In the previous section we have discussed many properties associated with \code{Iterable}s. Let's analyze now how these properties translate under dualisation and how they affect our new derived interface, the \code{Observable}.

First, moving from the observation that an \code{Iterable} is a getter of a getter, we can observe that the \code{Observable} plays exactly the opposite role, that is, a setter of a setter. The type \code{Observer :: a -> IO ()} represents, in fact, the essence of a setter function, whereas the \code{Observable} consists in nothing more than the simple task of applying the observer function to itself, producing a setter of setters. 

While the discussion about \code{Iterable}'s covariance was quite intuitive, things get a little bit more complicated when analyzing \code{Observable}s. Referring back to our previous example involving cokes and drinks, we can now think of the \code{Observer} as a a recycling machine:

\hminted
                                Coke <: Drink
                    -----------------------------------
                    RecyclingM Drink <: RecyclingM Coke
\end{minted}


Our intuition tells us that this time, a recycling machine that can only handle coke cannot be used in place of one that needs to handle any type of drinks, as it would fail at its task whenever a drink that is not a coke is fed into it. On the other hand, a recycling machine that works for any type of drink can be safely used in place of one that needs to handle cokes. This intuition bounds \code{Observer} and its base type \code{a} by a contravariant relation:

\hminted
                                   A <: B
                      ----------------------------------
                          A -> IO () <: B -> IO ()
\end{minted}


A more theoretical take on the matter involves the notion of type's positivity and negativity: we can interpret a function of type \code{f :: a -> b} as a way for us to produce a value of type \code{b}. In this context, \code{b} is considered to be positive with respect to the type \code{a -> b}. On the other hand, in order to apply the function, we are going to need a value of type \code{a}, which we will need to get from somewhere else; \code{a} is therefore considered to be negative w.r.t. the function type, as the function introduces a need for this value in order to produce a result. The point of this distinction is that positive type variables introduce a covariant relation between base and function type whereas negative type variables introduce a contravariant relation.

Analyzing \code{Iterable} within this framework is easy, the \code{Iterator} function contains a single type parameter found in a positive position, therefore resulting in a covariant relation; being the \code{Iterable} the result of applying the \code{Iterator} function to itself, we again result in a covariant relation w.r.t. the type parameter \code{a}.

The \code{Observer} function, on the contrary, introduces a need for a value of type \code{a}, resulting in a contravariant relation w.r.t. \code{a}. Again, the \code{Observable} function is the result of applying \code{Observer} to itself; surprisingly, this results in \code{a} being in a positive position. The intuition is easily understood by thinking about the rules of arithmetic multiplication: \code{a} is in negative position w.r.t. the \code{Observer} function, whereas the \code{Observer} is in negative position w.r.t. the \code{Observable}. This leads to \code{a} being negated twice, ultimately resulting in a positive position within the \code{Observable} function. 

\haskellcode[firstline=3,lastline=17]{src/polarity.hs}

Before we formalize this claim, let's convince ourselves that \code{Observable}s effectively produces a value of type \code{a} by looking at an example:

\haskellcode[firstline=21]{src/polarity.hs}

It is clear from this implementation that \code{randomValueObs} indeed produces a value of type \code{Int}, whereas the \code{Observer} introduces a need for such value in order to be applied. For more details on the positivity and negativity of functions and type variables, see \cite{pos-neg}\cite{pos-neg2}.

Just as we did with \code{Iterable/Iterator}, we can formally prove the covariant and contravariant relations between \code{Observable/Observer} and their base type \code{a} by instantiating them to \code{Functor} and \code{Contravariant (Functor)} respectively, as well as proving the laws.

\haskellcode{src/observable_functor.hs}

The reader acquainted with functional programming will easily see the resemblance between the \code{Observable} type and a CPS function (See Appendice \ref{app:a}).

\haskellcode{src/obs_cont.hs}

It is clear how an \code{Observable} is nothing more than a special case of a CPS function where the result type \code{r} is instantiated to \code{IO ()}. To convince ourselves of this equivalence, let's think about the definition of a CPS function, i.e. a suspended computation which, given another function - the continuation - as argument, will produce its final result. This definition suits perfectly the idea behind \code{Observable} discussed in Section \ref{subsec:rx}: a function which will do nothing - i.e. is suspended - until it is subscribed to by an \code{Observer}.

A continuation, on the other hand, represents the future of the computation, a function from an intermediate result to the final result\cite{newbern-monads}; in the context of \code{Observable}s, the continuation represents the \code{Observer}, a function specifying what will happen to a value produced by the \code{Observable}, whenever it will become available, that is, whenever it will be pushed into the \code{Observer}. Since a continuation can be called multiple times within the surrounding CPS context, it is easy to see how this mathematical concept allows us to deal with multiple values produced at different times in the future.

We can prove our claim by implementing the \code{Observable} interface using Haskell's Continuation Monad Transformer and observing how the unwrapping function \code{runContT} effectively hands us back our original type:

\hminted
newtype ContT r m a :: * -> (* -> *) -> * -> *
runContT :: ContT r m a -> (a -> m r) -> m r

type Observable a = ContT () IO a
runContT :: ContT () IO a -> (a -> IO ()) -> IO ()
\end{minted}


This equivalence is very important as it allows us to claim an instance of Applicative Functor and Monad for our derived type, \code{Observable}. These instances are inherited for free from the continuation monad, sparing us the burden of implementing them and proving all related laws.

\hminted
instance Applicative (ContT r m) where
    pure x  = ContT ($ x)
    f <*> v = ContT $ \c -> runContT f $ \g -> runContT v (c . g)

instance Monad (ContT r m) where
    return x = ContT ($ x)
    m >>= k  = ContT $ \c -> runContT m (\x -> runContT (k x) c)
\end{minted}
%$


\section{Termination and Error Handling}
\label{termerr}

We began this chapter by progressively simplifying the \code{Iterable}'s interface in order to derive a type that would theoretically represent its very essence. One of the most important steps was setting aside concerns regarding termination an error handling of a collection. We are now going to reshape our reactive interfaces in order to address these concerns and appropriately describe the potential side effects directly in the types. 

Informally, an \code{Observable} stream might not only produce one or more values, but it might gracefully terminate at a certain point in time or throw an exception and abruptly terminate whilst processing values. A more appropriate type for \code{Observer} is then the following:

\haskellcode{src/observable_compl_err.hs}

Just as with \code{Iterable}, the introduction of \code{Either SomeException} allows us to express that the \code{Observer} can handle unexpected exceptions, while the \code{Maybe} reflects the possibility for a stream to end and propagate no more values. 

Unfortunately, this type is very hard to read as well as understand for someone new to the topic. Looking at the matter from a functionality point of view, what we really need is for our CPS function - i.e. the \code{Observable} - to accept two additional continuations, one dealing with completion and one with exceptions. We can achieve this by first noticing that our type is nothing more than a coproduct - the same that we introduced previously for \code{Iterable} - of three base types: \code{a + SomeException + ()}. By utilizing the notion of product - the dual of coproduct - we can split the function handling the initial type into three different ones. This brings us to the final version of our reactive interfaces for push-base collections\footnote{The two definitions are equivalent also from an implementation point of view, the first simulating the second though the use of pattern matching.}

\haskellcode{src/observable_newtype.hs}

The \code{Observable} is now a special version of a CPS function accepting three continuation functions - embedded inside the \code{Observer} -, one for each effect an \code{Observable} can propagate: value, termination or exception.

\section{Formalizing Observables}
\label{obscont}

In section \ref{sec:applyingduality} we have shown how the essential type for \code{Observable} is effectively nothing more than a particular instance of the continuation monad. In this section we are going to explore this relation in further detail, introducing a notation which will help us keep track of the changes we will make to the original\code{Observable} type, ultimately showing how the resulting interface - that will be used in our final library - will consist in nothing more than a modified version of a CPS function.

We are going to start from the notion that \code{Observable} is, at its essence, nothing more than a setter of setters, the result of applying the \code{Observer} function to itself. We can than express the \code{Observer} as a function \code{(!)}\footnote{Regard the code used in this explaination as pseudo-Haskell.} that negates its type argument and results in a side-effectful computation.

\hmint |!a :: a -> IO () |

When we apply the function to itself - i.e. substitute \code{a} for \code{!a} - we obtain our first definition of \code{Observable}, a CPS function that instantiates the result to \code{IO ()}.

\hmint |!!a :: (a -> IO ()) -> IO () |

As we have seen in the previous section, this definition is not expressive enough when we want to make explicit all the effects that are involved when dealing with push-base collections. It is therefore necessary to deviate from the standard definition of continuation and replace the inner application of \code{(!)} with a new function \code{(?)}, whose type embed the involved effects:

\hminted
?a  ::  Either Error (Maybe a) -> IO () -- termiantion and error handling
!?a :: (Either Error (Maybe a) -> IO ()) -> IO ()
\end{minted}


Note how this definition is equivalent to the one used in the previous section, where we used the notion of products to unwrap the \code{(?)} function into three different continuation, each addressing one of the possible effects. 

In the next chapter we are going to further modify this definition with the inclusion of a cancellation mechanism.

This newer version of \code{Observable} is still implementable as an instance of the continuation monad, as the code below shows.

\haskellcode[firstline=5,lastline=16]{src/observable_events.hs}

The code above uses a slightly different approach to expressing the three types of side effects an \code{Observer} has to deal with. Insead of using \code{Either} and \code{Maybe} from Haskell's libraries, we utilize our own custom datatype \code{Event}, a coproduct of values of type \code{a + SomeException + ()}; although the two definitions are equivalent in every aspect, the adopted one offers more clarity in terms of code readability.

At this point we have all the necessary tools to create and run an \code{Observable}.

\haskellcode[firstline=18]{src/observable_events.hs}

Notice how, being a CPS function, an \code{Observable} only pushes values once subscribed to and acts as a suspended computation otherwise. 

The code above, being a toy example, fails to show some fundamental properties associated with this new interface; in particular, it fails to show how \code{Observable}s can actually handle asynchronous sources of data. The following snippet of code contains a more realistic and meaningful example of an \code{Observable} handling keyboard presses, asynchronous events by nature: whenever the user presses a key, an event containing the corresponding character is propagated to the \code{Observer} and will eventually be printed on the command line. It is worth noticing how our basic implementation of \code{Observable} based on continuations works just as well as a full blown one in terms of its core capability of handling asynchronous data.

\haskellcode[firstline=17]{src/create_run_demo.hs}

This example brings us to the following observation: \code{Observable}s are capable of handling asynchronous data sources, yet the means by which the data is handled are not asynchronous by default. This is a common misconception and source of much confusion among the community: the \code{Observable} interface is not opinionated w.r.t. concurrency and therefore, by default, synchronously handles it's incoming data, blocking the next incoming events whilst processing the current one. This behavior is not fixed though: as we will see in Chapter \ref{chap:outoftherabbithole}, it is possible to make use of \code{Scheduler}s to orthogonally introduce concurrency in our reactive systems, altering the control flow of the data processing allowing the user to dispatch the work to other threads.

At this point in the discussion we have arrived to a working implementation of a push based collection purely derived from mathematical and categorical concepts such as duality and continuations. In spite of being very insightful for theoretical discussions on the properties and relations of \code{Observable} and the continuation monad, this implementation of the reactive types is impractical in the context of a full fledged API. In the next Chapter we will take the necessary steps to build the bridge between theory and practice, providing a reference implementation of \code{Observable} more adapt to be utilized in real world applications.

For a full implementation of a Reactive Library based on the ideas presented in this section, aimed at highlighting the strong connection between \code{Observable} and other already existing functional structures from which it composes, see Appendice \ref{app:b}. 